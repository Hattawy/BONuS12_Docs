%
%---Beginning of Document---
%
\documentclass[12pt]{article}
\usepackage[dvips]{color,graphics}
\usepackage{longtable}
\usepackage{hyperref}
\usepackage[pdftex]{graphicx}
\usepackage[absolute,overlay]{textpos}
\setlength{\TPHorizModule}{1mm}
\setlength{\TPVertModule}{1mm}
\usepackage{rotating}
%
\textwidth  6.5in
\textheight 9.0in
\topmargin 0.0 in
\headheight 0.0in
\headsep 0.0in
\oddsidemargin 0in
\evensidemargin 0in
\parsep 0 in
%\parindent 0in
\pagenumbering{arabic}
%
\setcounter{secnumdepth}{5}
\setcounter{tocdepth}{5}
%
\begin{document}

\title{Run Group F RTPC Calibration Constants}

\vskip 0.5cm

\author{M. Hattawy and S.E. Kuhn, Old Dominion University\\[0.1ex]
{\it bonus12\_db.tex}}


\date \today
%
\maketitle

This document details the calibration constants for the BONuS12 Radial Time Projection Chamber (RTPC) system. They are used for gain calibration of all 17280 pads, and for the time-to-track position conversion.

\section*{}

\section{RTPC Calibration Constants}


%%%%%%%%%%%%%%%%%%%%%%%%%%%%%%%%%%%%%%%%%%%%%%%%%%%%%
\begin{table}[htbp]
\begin{center}
\begin{tabular} {|l|l|} \hline
Database Entry                                      & Description \\ \hline
/calibration/rtpc/gain\_balance           & Relative gain of all 17280 pads \\ \hline
/calibration/rtpc/time\_offsets           & Time offsets to determine actual drift time \\ \hline
/calibration/rtpc/time\_parms              & Parameters for drift time to hit position conversion \\ \hline
\end{tabular}
\caption{Overview table for the RTPC calibration constants.}
\label{rtcp-parms}
\end{center}
\end{table}
%%%%%%%%%%%%%%%%%%%%%%%%%%%%%%%%%%%%%%%%%%%%%%%%%%%%%

Table~\ref{rtcp-parms} provides an overview of the calibration constants associated 
with the custom Radial Time Projection Chamber (RTPC) that will be used during Run Group F
(Spring 2020). The 40 cm-long RTPC consists of a cylindrical cathode at 3 cm radius, surrounded by a 4 cm 
wide (radially) drift region. Drift electrons released from slow-moving protons (spectators in the reaction 
D$(e,e^\prime p_S)X$) drift to the outer radius (7 cm) of this region along curved paths (due to the 5 T
magnetic field in the CLAS12 solenoid) and are amplified in 3 layers of GEMs spaced 3 mm apart in radius. 
The final electrons are detected on an outer padboard of 17280 pads arranged in 180 rows in $\phi$ and
96 columns in z. Each pad is connected to a DREAM chip FEU from the CLAS12 MVT and the signal is
digitized as a sequence of pulse heights, 120 ns apart. This information is used to determine the mean arrival
time of the electrons and the total charge, which is proportional to the energy loss by the detected proton.

The constants fall into two 
different categories, those associated with the gain calibration 
and those associated with the timing calibration. A few more constants refer to the physical orientation 
of the RTPC in space and are discussed in a separate geometry document.

The CLAS12 calibration database (ccdB) is organized into columns based on {\it sector}, 
{\it layer}, and {\it component}. The RTPC is a hermetic cylinder with $2 \pi$ azimuthal
acceptance, located in the CLAS12 Central Detector and therefore is not
divided into sectors, Hence, the {\it sector} is always assigned as 1. (Similarly, the {\it order} 
variable is always 0).
The {\it layer} variable is used to design the number of a pad within a row (the column), from 1 through 96.
(This determines the pad position along the beam direction, $z$, for each of the 4 mm long pads).
Finally, the {\it component} variable
corresponds to the row of pads, from 1 through 180, where each row covers $2^\circ$ in $\phi$.
In each of the subsections below, full details on each of the RTPC database calibration
constants listed in Table~\ref{rtcp-parms} are provided.    

   \section {Gain calibration constants}
 The ionization electron collection system of the RTPC has 17280 readout pads. 
 The gain of each pad is the ratio between the energy deposited within the 
 track segment for which the ionization electrons will drift to that pad, and 
 the output signal recorded value. The energy loss of a particle along its 
 track in the RTPC, $\small{\frac{dE}{dX}}$, can be calculated from the 
 collected ADCs as: \begin{equation}
 \left\langle \frac{dE}{dX} \right\rangle= \frac{\sum\limits_{i} \frac{ADC_{i}}{Gi}}{vtl},
\end{equation}
where the sum runs over a set of pads contributing to a track. $ADC_{i}$ is the 
recorded amplitude in each pad $i$, and $G_{i}$ is its gain. The variable $vtl$ is the 
      total visible length of the track in the region viewed by the set of 
      pads. Corresponding to the 17280 readout pads for the BONuS12 RTPC, we 
      will need a total of 17280 constants, initially each set to 1.0.
    ~\\

       
We will extract the gains using two techniques. The first 
      one is by comparing the experimental recorded $\small{\frac{dE}{dX}}$ to 
      the expected values calculated from the Bethe-Bloch formula. The second 
      technique is based on comparing the experimental ADCs to the GEANT4 
      simulated ones, track by track. We can refine and cross-check this second 
      method comparing the ADCs of each pad to the average ADCs of other, 
      adjacent pads in the same track. These parameters will be extracted from 
      the elastic $^1$H$(e,e^\prime p)$ measurements at the beginning of the 
      BONuS12 run (at 2.2 GeV beam energy). The relevant details on the RTPC 
      gain calibration database are given in Table~\ref{gain}.

%%%%%%%%%%%%%%%%%%%%%%%%%%%%%%%%%%%%%%%%%%%%%%%%%%%%%
\begin{table}[htbp]
\begin{center}
\begin{tabular} {|c|c|c|c|} \hline
Name      & \multicolumn{3}{|l|} {gain\_balance} \\ \hline
Full path & \multicolumn{3}{|l|} {/calibration/rtpc/gain\_balance} \\ \hline
Rows       & \multicolumn{3}{|l|} {17280} \\ \hline
Columns  & \multicolumn{3}{|l|} {4} \\ \hline
Column info  & \# & type   & name \\ \hline
             & 0 & int   & {\it sector} \\ \cline{2-4}
             & 1  & int   & {\it layer} \\ \cline{2-4}
             & 2 & int   & {\it component} \\ \cline{2-4}
             & 3 & double & {\it gain} \\ \cline{2-4} \hline
\end{tabular}
\caption{RTPC gain calibration database information.}
\label{gain}
\end{center}
\end{table}


   
   \section {Time offset constants}
 These constants provide the relative timing between the signals on each pad 
 and the vertex time of the trigger electron. They are $T_{lat}$, $T_{prop}$ 
 and $T_{rew}$. Initially they will be set as $T_{lat}=~8\mu s$, $T_{prop} = 1 
 \mu s$ and $T_{rew}=~9.6\mu s$.
~\\
~\\
$T_{lat}$ is the latency of the trigger, i.e. how long after a typical electron 
leaves the vertex does the trigger signal 
reach the DREAM chips. This can be calculated from the typical electron flight time to the EC
(roughly 15 ns - not critical given the RTPC time resolution), the delay between the EC hit time
and the trigger supervisor trigger signal output, and the known delays through the BEU and FEU
electronics for a trigger of the FEU (about 4.14 $\mu$s).
$T_{prop}$ refers to the time between the electron vertex time and the mean (weighted) time of the first output signal from the DREAM pulse shaper, consisting of the sum of all delays (including the
travel time of a proton into the RTPC, 
the drift time from the first GEM to the readout pad, the cable delays, and the intrinsic DREAM signal width), plus any other offsets
in the sample storage and conversion process. (Notice that the actual drift time in the drift region
itself is accounted for in the next section).
This time will be taken from the data, by looking at the
distribution of the SHORTEST drift times for each event. This distribution will have a random (constant) accidental
background, with a spike on top for real coincident, in-time spectator protons. $T_{prop}$ will be set such
that this spike appears at time zero.
Finally, $T_{rew}$ is the ``rewind time'', i.e. how much before the trigger time do we 
want to start reading out the stored samples in the DREAM chips. Our first 
guess of this parameter will be refined from the commissioning runs for RG-C. 

The relevant details on the RTPC time offset constants are given in 
Table~\ref{time_off}.

%%%%%%%%%%%%%%%%%%%%%%%%%%%%%%%%%%%%%%%%%%%%%%%%%%%%%
\begin{table}[htbp]
\begin{center}
\begin{tabular} {|c|c|c|c|} \hline
Name      & \multicolumn{3}{|l|} {time\_offsets} \\ \hline
Full path & \multicolumn{3}{|l|} {/calibration/rtpc/time\_offsets} \\ \hline
Rows       & \multicolumn{3}{|l|} {1} \\ \hline
Columns  & \multicolumn{3}{|l|} {3} \\ \hline
Column info  & \# & type   & name \\ \hline
             & 0 & double   & {\it T\_lat } \\ \cline{2-4}
             & 1  & double   & {\it T\_prop } \\ \cline{2-4}
             & 2 & double   & {\it T\_rew } \\ \cline{2-4}  \hline
\end{tabular}
\caption{RTPC time offset database information.}
\label{time_off}
\end{center}
\end{table}


  
   \section{Drift paths and Drift speed constants}
 
    In a TPC, the electrons released in an ionization drift towards the readout 
  board following their drift paths under the effect of the applied 
  electromagnetic field. The drift speed depends also on the gas mixture, pressure and
  temperature. The 
  recorded drift time with the known drift speed of the electrons provides 
  information on how far away from the outside surface the initial ionizations happened in the drift region, 
  leading to reconstruct the original radial distance from the beam axis of ionizations in the drift 
  region. Furthermore, due to the Lorentz angle (the curvature of ionization tracks in the
  presence of the 5 T solenoid field), we also have the account for the displacement in $\phi$ 
  of the position of a pad with a recorded signal and the location of the primary ionization event.
  
  Based on our Garfield++ simulation, the best fit of the radial and the azimuthal 
  position  of a reconstructed hit can be formulated as a function of the elapsed drift time $t$ 
  between the ionization and the arrival of the drift electrons at the anode as: 

\begin{equation}
    r(t)= \frac{-\sqrt{a^2 + 4bt} + a + 14b}{2b}
\end{equation}
  
\begin{equation}
   \Delta \phi (r)= c(7-\frac{r}{10}) + d(7-\frac{r}{10})^2
\end{equation}
  
where $a$ and $b$ are parameters set initially by a fit in Garfield++
based on the gas, and the electromagnetic field. Because of E \& B fields 
non-uniformity, $a$ and $b$ depend on $z$ position along the detector. Hence, 
to reconstruct tracks in three dimensions from the measured arrival times on 
each pad of a track, we will need four parameters, $a$, $b$, $c$, $d$, each of 
which are functions (4th order polynomials) of $z$ (in mm) along the detector 
as follows (with preliminary values from the GARFIELD++ simulation), for a 
total of 20 calibration constants:

\begin{equation}
a\_t = a1 * z^4 + a2 * z^3 + a3 *z^2 + a4 * z + a5
\end{equation}
with\\
a1 = -2.48491E-4\\
a2 = 2.21413E-4\\
a3 = -3.11195E-3\\
a4 = -2.75206E-1\\
a5 = 1.74281E3\\

\begin{equation}
b = b1 * z^4 + b2 * z^3 + b3 *z^2 + b4 * z + b5
\end{equation}
with\\
b1 = 2.48873E-5\\
b2 = -1.19976E-4\\
b3 = -3.75962E-3\\
b4 = 5.33100E-2\\
b5 = -1.25647E2\\


\begin{equation}
c = c1 * z^4 + c2 * z^3 + c3 *z^2 + c4 * z + c5
\end{equation}
with\\
c1 = -3.32718E-8\\
c2 = 1.92110E-7\\
c3 = 2.16919E-6\\
c4 = -8.10207E-5\\
c5 = 1.68481E-1\\

\begin{equation}
d = d1 * z^4 + d2 * z^3 + d3 *z^2 + d4 * z + d5
\end{equation}
with\\
d1 = -3.23019E-9\\
d2 = -6.92075E-8\\
d3 = 1.24731E-5\\
d4 = 2.57684E-5\\
d5 = 2.10680E-2\\


 The relevant details on the RTPC time parameters database are given in 
 Table~\ref{time_parms}.

%%%%%%%%%%%%%%%%%%%%%%%%%%%%%%%%%%%%%%%%%%%%%%%%%%%%%
\begin{table}[htbp]
\begin{center}
\begin{tabular} {|c|c|c|c|} \hline
Name      & \multicolumn{3}{|l|} {time\_parms} \\ \hline
Full path & \multicolumn{3}{|l|} {/calibration/rtpc/time\_parms} \\ \hline
Rows       & \multicolumn{3}{|l|} {4} \\ \hline
Columns  & \multicolumn{3}{|l|} {5} \\ \hline
Column info  & \# & type   & name \\ \hline
             & 0 & double   & {\it const\_4th\_exp} \\ \cline{2-4}
             & 1  & double   & {\it const\_3rd\_exp } \\ \cline{2-4}
             & 2 & double   & {\it const\_2nd\_exp} \\ \cline{2-4}
            & 3 & double & {\it const\_1st\_exp} \\ \cline{2-4}                 
            & 4 & double & {\it const} \\ \cline{2-4} \hline
\end{tabular}
\caption{Time parameters database information.}
\label{time_parms}
\end{center}
\end{table}



 
These constants will be recalibrated using the elastic $^1$H$(e,e'p)$  data 
from the commissioning runs by optimizing the fit between the reconstructed 
proton tracks and the inferred ones from the electron kinematics.

In practical terms, it will be useful to normalize all drift times to the maximum drift time,
$T_{max}$, for an ionization event that happens right at the position of the cathode. Therefore,
we will need one additional calibration constant for $T_{max}$. This will be set
initially again from the GARFIELD++ simulation ($T_{max} \approx 6 \mu s$) and then 
adjusted by a fit to the elastic data. Possible small shifts over time in  $T_{max}$ 
(due to changing gas composition, temperature, and pressure) will be monitored
by the custom Drift time Monitoring System (DMS) that has been constructed for BONuS12.
This will allow us to adjust the overall reconstruction from drift time to ionization point run by run
without having to recalibrate all remaining 20 parameters.


\end{document} 

