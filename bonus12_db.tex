%
%---Beginning of Document---
%
\documentclass[12pt]{article}
\usepackage[dvips]{color,graphics}
\usepackage{longtable}
\usepackage{hyperref}
\usepackage[pdftex]{graphicx}
\usepackage[absolute,overlay]{textpos}
\setlength{\TPHorizModule}{1mm}
\setlength{\TPVertModule}{1mm}
\usepackage{rotating}
%
\textwidth  6.5in
\textheight 9.0in
\topmargin 0.0 in
\headheight 0.0in
\headsep 0.0in
\oddsidemargin 0in
\evensidemargin 0in
\parsep 0 in
%\parindent 0in
\pagenumbering{arabic}
%
\setcounter{secnumdepth}{5}
\setcounter{tocdepth}{5}
%
\begin{document}

\title{BONuS12 Calibration Constants}

\vskip 0.5cm

\author{M. Hattawy and S. Kuhn\\ Old Dominion University}

\date \today
%
\maketitle

This document details the calibration constants for the BONuS12 system as 
follows:

\section*{}

   

   \section {Gain calibration constants}
 The ionization electron collection system of the RTPC has 17280 readout pads. The gain of 
      each pad is the ratio between the 
      %%%SEK
      energy deposited within the track segment
      for which the ionization electrons will drift to that pad, 
      %%%SEK
      and the output signal
      recorded value.      %%%SEK
   The energy loss of a particle along its track in the RTPC,
$\small{\frac{dE}{dX}}$, can be calculated from the collected ADCs as: 
      \begin{equation}
 \left\langle \frac{dE}{dX} \right\rangle= \frac{\sum\limits_{i} \frac{ADC_{i}}{Gi}}{vtl},
\end{equation}
where the sum runs over all the pads contributing to a track. $ADC_{i}$ is the 
recorded amplitude in each pad $i$, and $G_{i}$ is its gain. The variable $vtl$ is the 
      total visible length of the track in the active drift volume. 

Corresponding to the 17280 readout pads for the BONuS12 RTPC, we will need a
      %%%SEK
   total of 17280 constants, initially each set to 1.0.
    ~\\

      %%%SEK
We will extract the gains using two techniques. The first 
      one is by comparing the experimental recorded $\small{\frac{dE}{dX}}$ to 
      the expected values calculated from the Bethe-Bloch formula. The second 
      technique is based on comparing the experimental ADCs to the GEANT4 
      simulated ones, track by track. We
      %%%SEK
       can refine and cross-check 
      %%%SEK
       this second method by 
      comparing the ADCs of each pad to the average ADCs of 
       %%%SEK
	other, adjacent pads in 
      %%%SEK
      the same track. These parameters will be extracted from the elastic 
      $^1$H$(e,e'p)$ measurements at the beginning of the BONuS12 run (at 2.2 GeV beam energy). 

   
   \section {Time offset constants}
  
      %%%SEK
	These constants provide the relative timing between the signals on each pad
	and the vertex time of the trigger electron. They are
      %%%SEK
   $T_{lat}$ and $T_{rew}$. Initially they will be set as $T_{lat}=~8\mu s$ and 
   $T_{rew}=~9.6\mu s$.
~\\
~\\
$T_{lat}$ is the latency of the trigger, i.e. how long after a typical electron 
leaves the vertex does the trigger signal %%%SEKneed to 
reach the DREAM chips. %%%SEK While 
$T_{rew}$ is the ``rewind time'', i.e. how much before the trigger time do we 
want to start reading out the stored samples in the DREAM chips. Our first 
guess of these two parameters will be refined from the commissioning runs for RG-C. 
   
  
   \section{Drift paths and Drift speed constants}
%%%SEK
    In a TPC, the electrons released in an ionization drift towards the readout 
  board following their drift paths under the effect of the applied 
  electromagnetic field. The drift speed depends also on the gas mixture, pressure and
  temperature. The 
  recorded drift time with the known drift speed of the electrons provides 
  information on how far away from the outside surface the initial ionizations happened in the drift region, 
  leading to reconstruct the original points of ionizations in the drift 
  region. Furthermore, due to the Lorentz angle (the curvature of ionization tracks in the
  presence of the 5 T solenoid field), we also have the account for the displacement in $\phi$ 
  of the position of a pad with a recorded signal and the location of the primary ionization event.
  
  Based on our Garfield++ simulation, the best fit of the radial and the azimuthal 
  position  of a reconstructed hit can be formulated as a function of the elapsed drift time $t$ 
  between the ionization and the arrival of the drift electrons at the anode as: 

\begin{equation}
    r(t)= \frac{-\sqrt{a^2 + 4bt} + a + 14b}{2b}
\end{equation}
  
\begin{equation}
   \Delta \phi (r)= c(7-\frac{r}{10}) + d(7-\frac{r}{10})^2
\end{equation}
  
where $a$ and $b$ are parameters set initially by a fit in Garfield++
based on the gas, and the electromagnetic field. Because of E \& B fields 
non-uniformity, $a$ and $b$ depend on $z$ position along the detector. 

   
   Hence, to reconstruct tracks in three dimensions from the measured arrival times on each
   pad of a track, we will need 
      %%%SEK
   four parameters, $a$, $b$, $c$, $d$, each of which are functions (4th order polynomials) of $z$ (in mm) along 
   the detector as follows (with preliminary values from the GARFIELD++ simulation), for a total of 20
   calibration constants:

\begin{equation}
a\_t = a1 * z^4 + a2 * z^3 + a3 *z^2 + a4 * z + a5
\end{equation}
with\\
a1 = -2.48491E-4\\
a2 = 2.21413E-4\\
a3 = -3.11195E-3\\
a4 = -2.75206E-1\\
a5 = 1.74281E3\\

\begin{equation}
b = b1 * z^4 + b2 * z^3 + b3 *z^2 + b4 * z + b5
\end{equation}
with\\
b1 = 2.48873E-5\\
b2 = -1.19976E-4\\
b3 = -3.75962E-3\\
b4 = 5.33100E-2\\
b5 = -1.25647E2\\


\begin{equation}
c = c1 * z^4 + c2 * z^3 + c3 *z^2 + c4 * z + c5
\end{equation}
with\\
c1 = -3.32718E-8\\
c2 = 1.92110E-7\\
c3 = 2.16919E-6\\
c4 = -8.10207E-5\\
c5 = 1.68481E-1\\

\begin{equation}
d = d1 * z^4 + d2 * z^3 + d3 *z^2 + d4 * z + d5
\end{equation}
with\\
d1 = -3.23019E-9\\
d2 = -6.92075E-8\\
d3 = 1.24731E-5\\
d4 = 2.57684E-5\\
d5 = 2.10680E-2\\

     %%%SEK
 
These constants
will be recalibrated using the elastic 
      $^1$H$(e,e'p)$  data from the commissioning runs by optimizing the fit between 
the reconstructed proton tracks and the inferred ones from the electron kinematics.

In practical terms, it will be useful to normalize all drift times to the maximum drift time,
$T_{max}$, for an ionization event that happens right at the position of the cathode. Therefore,
we will need one additional calibration constant for $T_{max}$. This will be set
initially again from the GARFIELD++ simulation ($T_{max} \approx 6 \mu s$) and then 
adjusted by a fit to the elastic data. Possible small shifts over time in  $T_{max}$ 
(due to changing gas composition, temperature, and pressure) will be monitored
by the custom Drift time Monitoring System (DMS) that has been constructed for BONuS12.
This will allow us to adjust the overall reconstruction from drift time to ionization point run by run
without having to recalibrate all remaining 20 parameters.


\end{document} 

