%
%---Beginning of Document---
%
\documentclass[12pt]{article}
\usepackage[dvips]{color,graphics}
\usepackage{longtable}
\usepackage{hyperref}
\usepackage[pdftex]{graphicx}
\usepackage[absolute,overlay]{textpos}
\setlength{\TPHorizModule}{1mm}
\setlength{\TPVertModule}{1mm}
\usepackage{rotating}
%
\textwidth  6.5in
\textheight 9.0in
\topmargin 0.0 in
\headheight 0.0in
\headsep 0.0in
\oddsidemargin 0in
\evensidemargin 0in
\parsep 0 in
%\parindent 0in
\pagenumbering{arabic}
%
\setcounter{secnumdepth}{5}
\setcounter{tocdepth}{5}
%
\begin{document}

\title{Radiological Safety Analysis Document for
BONuS12 in Hall B}


%\author{M. Hattawy and S.E. Kuhn, Old Dominion University\\[0.1ex]
%{\it bonus12\_db.tex}}


\date \today
%
\maketitle

This Radiological Safety Analysis Document (RSAD) will iden-
tify the general conditions associated with the BONuS12 run using
a newly built RTPC the standard CLAS12 detector in Hall B and
the controls associated with regard to production, movement, or
import of radioactive materials. 

\section{Description}
The BONuS12 experiment will take place in the winter of 2018 or the
Spring of 2019 in experimental Hall B using a newly built Radial Time 
Projection Chamber(RTPC) and the CLAS12 detector. CLAS12 is a multi-purpose
detector system based on a toroidal (forward detector) and a solenoid (central 
detector) magnet. The detector system includes Cherenkov Counters,
Drift Chambers, Scintillator Counters, Silicon-strip detectors, Micro-mega
gas detectors, and Calorimeters. The RTPC is a small cylindrical GEM de-
tector of 40 cm long and about 8 cm of radius, surrounding the target. The
RTPC will be tuned to detect only low momentum protons. The target for
BONuS12 will be a tiny volume of 6 atm (absolute) deuterium or hydrogen
gas at room temperature, located at the center of the solenoid, which is also
the center of the hall. Beams of various energies, from 2.2 up to 10.5 GeV, and
maximum beam currents up to 200 nA will be used for the experiment.
The whole experiment includes two parts. The first part is calibration
runs using 2.2 GeV beam, which includes two hours of empty target, one day
on hydrogen and one day on deuterium target. The second part run with
200 nA beam at 10.5 GeV, which includes 35 days on deuterium target and
four days on hydrogen target, plus one day of calibration runs using empty
target. The peak nucleon luminosity of BONuS12 (not including the end window
of the beamline) is $2\times 10^{34}/cm^{2}/s$, which is only 20\% of the 
designed CLAS12 luminosity. In order to calibrate the drift velocity of the 
drift-gas used by the RTPC, a drift chamber (20cm $\times$ 20cm $\times$ 15 cm 
box shape) will be used to measure the drift velocity. This device is not 
placed in the beamline, but somewhere in
the hall using two standard (2 $\mu$Ci) $^{90}$Sr radiation sources. The 
$^{90}$Sr radiation sources will be well shielded. Before placing this device 
into the hall, this device shall be reviewed and approved by the Radiation 
Control Department.

\section{Summary and Conclusions}
The experiment is not expected to produce significant levels of radiation at 
the site boundary. However, it will be periodically monitored by
the Radiation Control Department to ensure that the site boundary goal is
not exceeded. The main consideration is the manipulation and/or handling
of target(s) or beamline hardware. As specified in Sections 4.2 and 7, the
manipulation and/or handling of target(s) or beamline hardware (potential
radioactive material), the transfer of radioactive material, or modifications
to the beamline after the target assembly must be reviewed and approved by
the Radiation Control Department. Adherence to this RSAD is vital.

\section{Calculations of Radiation Deposited in the
Experimental Hall (the Experiment Operations Envelope)}

The radiation budget is the amount of radiation that is expected at the
site boundary as a result of a given experiments. This budget may be specified 
in terms of mrem at the site boundary or as a percentage of the Jefferson Lab 
design goal for dose to the public, which is 10 mrem per year. The Jefferson 
Lab design goal is 10\% of the DOE annual dose limit to the public, and
cannot be exceeded without prior written consent from the Radiation Control 
Department Head, the Director of Jefferson Lab, and the Department of Energy.

Calculations of radiation in the hall has been carried out using FLUKA.
To simplify the task, 42 days of maximum luminosity has been used in this
calculation. In the FLUKA program, the RTPC holder, beam pipes and their
end windows, the RTPC, and the CLAS12 solenoid have been included. Here
are the FLUKA simulation results.
\begin{itemize}
   \item Prompt dose rate: prompt dose rate is low everywhere in the hall
outside the beam pipe. The hot area is in the downstream of the
target, which is below 10 rem/h.
\item Accumulated damage: after 42 days of 200 nA and 10.5 GeV beam, the
accumulated 1-MeV-neutron equivalent damage to silicon is less than
      10$^{11}$ neutrons/cm$^2$ outside the beam pipe. (The limit that a 
      silicon
      product starts to show damage is about several 10$^{13}$ 
      neutrons/cm$^{2}$ .)
\item Activation: after 42 days of running, the dose rate from activation at
30 minutes after the beam is shut down is about 0.03 mrem/h at the
target center, less than 0.001 mrem/h at 1 m away from the beam pipe.
\end{itemize}

      These calculated results will be verified during the experiment. The site
boundary radiation has not been calculated, but it is expected to be negligible
based on operational experience in Hall B. This will be verified by using
active monitors at the Jefferson Lab site boundary to keep up with the dose
for the individual setups from Hall B and the other Halls. If it appears that
the radiation budget will be exceeded, the Radiation Control Department
will require a meeting with the experimenters and the Head of the Physics
Division to determine if the experimental conditions are accurate, and to
assess what actions may reduce the dose rates at the site boundary. If the
site boundary dose approaches or exceeds 10 mrem during any calendar year,
the experimental program will stop until a resolution can be reached.

\section{Radiation Hazards}
The following controls shall be used to prevent the unnecessary exposure
of personnel and to comply with federal, state, and local regulations, as well
as with Jefferson Lab and the experimenter’s home institution policies.

\subsection{From Beam in the Hall}
When the Hall status is Beam Permit, there are potentially lethal con-
ditions present. Therefore, prior to going to Beam Permit, several actions
will occur. Announcements will be made over the intercom system notifying
personnel of a change in status from Restricted Access (free access to the
Hall is allowed, with appropriate dosimetry and training) to Sweep Mode.
All magnetic locks on exit doors will be activated. Persons trained to sweep
the area will enter by keyed access (Controlled Access) and search in all areas
of the Hall to check for personnel.
After the sweep, another announcement will be made, indicating a change
to Power Permit, followed by Beam Permit. The lights will dim and Run-Safe
boxes will indicate “OPERATIONAL” and “UNSAFE”. IF YOU ARE IN
THE HALL AT ANY TIME THAT THE RUN-SAFE BOXES INDICATE
UNSAFE, IMMEDIATELY HIT THE BUTTON ON THE BOX.
Controlled Area Radiation Monitors (CARMs) are located in strategic areas 
around the Hall and the Counting House to ensure that unsafe conditions
do not occur in occupiable areas.

\subsection{From Activation of Target and Beamline Components}
All radioactive materials brought to Jefferson Lab shall be identified to the
Radiation Control Department. These materials include, but are not limited
to, radioactive check sources (of any activity, exempt or non-exempt), 
previously used targets or radioactive beamline components, or previously used
shielding or collimators. The Radiation Control Department inventories and
tracks all radioactive materials onsite. The Radiation Control Department
will survey all experimental setups before experiments begin as a baseline for
future measurements.

The Radiation Control Department will coordinate all movement of used
targets, collimators, and shields. The Radiation Control Department will
assess the radiation exposure conditions and will implement controls as 
necessary based on the radiological hazards.
There shall be no local movement of activated target configurations without 
direct supervision by the Radiation Control Department. There is no
movement or change of target configurations in this experiment, except if the
RTPC or the experimental target cell should fail and need to be replaced, and 
after the end of the experiment.

No work is to be performed on beamline components, which could result
in dispersal of radioactive material (e.g., drilling, cutting, welding, etc.).
Such activities must be conducted only with specific permission and control
of the Radiation Control Department.

\section{Incremental Shielding or Other Measures to be Taken to Reduce 
Radiation Hazards}
After 42 days of beam time, the accumulated 1-MeV-neutron equivalent
damage outside the beam pipe is below 10$^{11}$ neutrons/cm$^2$ , which 
requires no extra shielding anywhere in the hall.

\section{Operations Procedures}
All experimenters must comply with experiment-specific administrative
controls. These controls begin with the measures outlined in the experiment's
Conduct of Operations Document, and also include, but are not limited to,
Radiation Work Permits, Temporary Operational Safety Procedures, and
Operational Safety Procedures, or any verbal instructions from the Radiation
Control Department. A general access RWP is in place that governs access
to Hall B and the accelerator enclosure, which may be found in the Machine
Control Center (MCC); it must be read and signed by all participants in the
experiment. Any individual with a need to handle radioactive material at
Jefferson Lab shall first complete Radiation Worker (RW I) training.
There shall be adequate communication between the experimenter(s) and
the Accelerator Crew Chief and/or Program Deputy to ensure that all power
restrictions on the target are well known. Exceeding these power restrictions 
may lead to excessive and unnecessary contamination, activation, and
personnel exposure.
No scattering chamber or downstream component may be altered out-
side the scope of this RSAD without formal Radiation Control Department
review. Alteration of these components (including the exit beamline itself)
may result in increased radiation production from the Hall and a resultant
increase in the site boundary dose.

\section{Decommissioning and Decontamination of Radioactive Components}
Experimenters shall retain all targets and experimental equipment brought
to Jefferson Lab for temporary use during the experiment. After sufficient 
decay of the radioactive target configurations, they shall be delivered to the 
experimenter's home institution for final disposition. All transportation shall 
be done in accordance with United States Department of Transportation 
Regulations (Title 49, Code of Federal Regulations) or International Air 
Transport Association regulations. In the event that the experimenter's home 
institution cannot accept the radioactive material due to licensing 
requirements, the experimenter shall arrange for appropriate fund transfers for 
disposal of the material. Jefferson Lab cannot store indefinitely any 
radioactive targets or experimental equipment.
The Radiation Control Department may be reached at any time
through the Accelerator Crew Chief (269-7050).


Approvals:

Radiation Control Department Head    ~~~~~~~~~~~~~~~~~~~~~~~~~~~~~~~~~~~~ Date



\end{document} 

